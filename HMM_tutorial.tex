\documentclass{article}
\usepackage{amsmath, amsfonts, amscd, amssymb, graphicx}

\title{Hidden Markov Models}
\numberwithin{equation}{section}

\newcommand{\len}{\mathrm{len}}
\newcommand{\argmax}{\operatornamewithlimits{argmax}}
\renewcommand{\P}{P}

\setlength{\parindent}{0in}

\newtheorem{problem}{Problem}

\begin{document}
\maketitle

Before we study Hidden Markov Models (HMM) it is necessary to begin
with an introduction to dynamic programming (DP).

\section{An Intro to Dynamic Programming}
An instructive example for the use of DP deals with finding the
optimal alignment between two DNA strands.  DNA consists of two
stands intertwined like vines bound together by hydrogen bonds.
These bonds, called bases, contain information about the organism.
There are four types of bases; adenine (abbreviated A), cytosine
(C), guanine (G) and thymine (T).

When studying the evolutionary origin and relations of different
species it is often helpful to compare the sequences of bases from
two different DNA strands. Suppose we wanted to compare two
sequences of DNA, $x$ and $y$, noting how many times $x_i=y_i$,
where $x_i$ is the $i$th element of x. Let $x=\{\textrm{C A G A T G
A G C A}\}$ and $y=\{\textrm{T C A A T G A A C A}\}$ then their
ungapped alignment is shown in figure \ref{ungapped}.

\begin{figure}[h]
\centering
\begin{tabular}{cccccccccc}
  C & A & G & A & T & G & A & G & C & A \\
   &  &  & $\mid$ & $\mid$ & $\mid$ & $\mid$ &  & $\mid$ & $\mid$ \\
  T & C & A & A & T & G & A & A & C & A \\
\end{tabular}
\caption{ungapped alignment} \label{ungapped}
\end{figure}

As can be seen in Figure \ref{ungapped} this alignment has 6
matches. If we allow gaps, denoted by '$-$', to be inserted into the
sequences then we can create more matches.  Figure \ref{gap} shows
us gapped alignment.

\begin{figure}[h]
\centering
\begin{tabular}{ccccccccccc}
$-$ & C & A & G & A & T & G & A & G & C & A \\
  & $\mid$ & $\mid$ & & $\mid$ & $\mid$ & $\mid$ & $\mid$ & & $\mid$ & $\mid$ \\
T & C & A & $-$ & A & T & G & A & A & C & A \\
\end{tabular}
\caption{gapped alignment} \label{gap}
\end{figure}

Since adding gaps changes the original sequence we will denote the
set of all gapped sequences of a sequence $x$ by $\mathcal{A}_x$. We
let $x\in\mathcal{A}_x$.  Suppose we are interested in finding the
optimal gapped alignment of two sequences $x$ and $y$ using the
scoring function $S$ given by

\[
S(x_i,y_i)=\left\{
\begin{array}{ll}
1 & \mbox{ if } x_i=y_i \\
-1 & \mbox{ if } x_i\neq y_i \\
-2 & \mbox{ if } x_i=- \mbox{ or } y_i=-
\end{array}
\right.
\]

Then the optimal alignment is given by solving
\begin{eqnarray}\label{system}
\max_{(\hat{x},\hat{y})\in\mathcal{A}_x\times\mathcal{A}_y} &&
\left\{\sum_i^{\len(\hat{x})}S(\hat{x}_i,\hat{y}_i)\right\}\nonumber
\\
\textrm{s.t.} &&  \len(\hat{x})=\len(\hat{y}) \nonumber
\end{eqnarray}

where $\len(x)$ denotes the number of elements in $x$.  Note that we
do not require $\len(x)=\len(y)$ but only
$\len(\hat{x})=\len(\hat{y})$.  The alignment in Figure \ref{gap} is
the solution to the system given by (\ref{system}) with a score of
3.

One way to solve this system for any given $x$ and $y$ is to check
the score of all possible gapped alignments by considering every
$\hat{x}$ and $\hat{y}$ satisfying the constraints in
(\ref{system}).  This method is obviously computationally
inefficient. Dynamic programming allows us to eliminate suboptimal
alignments by breaking solving subproblems of the original problem.

\subsection{Needleman-Wunsch algorithm}

The Needleman-Wunsch algorithm is a DP that solves (\ref{system}).
Given two sequences $x$ and $y$ let $\len(x)=n$ and $\len(y)=m$.  We
will let $F$ be an $(n+1)\times (m+1)$ matrix where F(i,j)
represents the score of the optimal alignment for
$x_1,x_2,\ldots,x_{i-1}$ and $y_1,y_2,\ldots,y_{j-1}$. Since we have
boundary issues with this definition of F we let $F(i,1)=-2(i-1)$
for $i=1,\ldots,n+1$ which gives us the score of aligning the
$x_1,\ldots,x_{i-1}$ with all gaps.  Likewise, $F(1,j)=-2(j-1)$ for
$j=2,\ldots,m+1$.  Since we have the first column and row of $F$ we
can find the rest of the entries by solving the recursive equation
given by

\[
F(i,j)=\max\left\{
\begin{array}{l}\label{recursion1}
F(i-1,j-1)+S(x_i,y_i) \nonumber \\
F(i-1,j)-2 \\
F(i,j-1)-2 \nonumber
\end{array}
\right. .
\]

This recursive equation for $F(i,j)$ helps us see if it is better to
align $x_{i-1}$ with $y_{i-1}$, $x_{i-1}$ with a gap, or $y_{i-1}$
with a gap.  Once the entire matrix has been generated we trace back
from the entry $F(n+1,m+1)$ to see which decisions were made to find
the max of each entry.  If $F(n,m)+S(x_{n+1},y_{m+1})$ was the max
then we know that in the optimal alignment $x_n$ was aligned with
$y_m$ and we next look at $F(n,m)$.   If $F(n,m+1)-2$ was the max
then $x_n$ was aligned with a gap and we look at $F(n,m+1)$ next. If
$F(m+1,n)-2$ was the max then $y_m$ was aligned with a gap and we
look at $F(m+1,n)$.  By continuing this process and sweeping back to
the beginning of the alignment we can find the optimal alignment.

In summary the Needleman-Wunsch algorithm is outlined in the
following steps.

\begin{enumerate}
\item Initialization: Define $(n+1)\times(m+1)$ matrix $F$.
    \begin{enumerate}
    \item $F(i,1)=-2(i-1)$ for $i=1,\ldots,n+1$
    \item $F(1,j)=-2(j-1)$ for $j=2,\ldots,m+1$
    \end{enumerate}
\item Recursion: Let
$F(i,j)=\max\{F(i-1,j-1)+S(x_i,y_i),F(i-1,j)-2,F(i,j-1)-2\}$ for
$i=2,\ldots,n+1$ and $j=2,\ldots,m+1$.
\item Trace back: Record where you came from starting at
$F(n+1,m+1)$.
\end{enumerate}


\subsubsection{Needleman-Wunsch Example}
For example, let $x=\{\textrm{T A C A T G C}\}$ and $y=\{\textrm{T T C A G C}\}$, e.g.,
\begin{center}
\begin{tabular}{ccccccc}
T & A & C &  A &  T &  G & C \\
$\mid$ &  & $\mid$ & $\mid$ & & $\mid$ & $\mid$\\
T & T & C & A & $-$ & G & C \\
\end{tabular}
\end{center}

We will use the scoring function $S$ given by

\[
S(x_i,y_i)=\left\{
\begin{array}{ll}
1 & \mbox{ if } x_i=y_i \\
-1 & \mbox{ if } x_i\neq y_i \\
-2 & \mbox{ if } x_i=- \mbox{ or } y_i=-
\end{array}
\right.
\]
and the Needleman-Wunsch algorithm to find the optimal alignment. 

First We will define the following DP matrix:

\begin{center}
\begin{tabular}{|c|rrrrrrrr|}\hline
              &      &{\tt t}    &{\tt a}      &{\tt c} &{\tt a}     &{\tt t}     &{\tt g} &{\tt c}\\ \hline% ttcac
               & & & & & & &  &\\
 \texttt{t} & & & & & & & &\\
 \texttt{t} & & & & & & & &\\
 \texttt{c} & & & & & & & &\\
 \texttt{a} & & & & & & &  &\\
 \texttt{g} & & & & & & &  &\\
 \texttt{c} & & & & & & &   &\\\hline
 \end{tabular}
\end{center}

Then completing the margins using the Needleman-Wunsch scoring: 

\begin{center}
\begin{tabular}{|c|rrrrrrrr|}\hline
              &      &{\tt t}    &{\tt a}      &{\tt c} &{\tt a}     &{\tt t}     &{\tt g} &{\tt c}\\ \hline% ttcac
               & 0   & -2        &-4          & -6         &-8          & -10  & -12 & -14\\
 \texttt{t} & -2 &  & & & & & & \\
 \texttt{t} & -4 & & & & & & & \\
 \texttt{c} & -6 & & & & & & & \\
 \texttt{a} & -8 & & & & & &  & \\
 \texttt{g} & -10 & & & & & &  &\\
 \texttt{c} & -12 & & & & & &   & \\\hline
 \end{tabular}
 \end{center}

And finishing the entire matrix:

\begin{center}
\begin{tabular}{|c|rrrrrrrr|}\hline
              &      &{\tt t}    &{\tt a}      &{\tt c} &{\tt a}     &{\tt t}     &{\tt g} &{\tt c}\\ \hline% ttcac
               & 0   & -2        &-4          & -6         &-8          & -10  & -12 & -14\\
 \texttt{t} & -2  & {\bf 1} & -1 & -3 & -5 & -7 & -9 & -11\\
 \texttt{t} & -4  & -1 & {\bf 0} & -2 & -4 & -4 & -6 & -8 \\
 \texttt{c} & -6  & -3 & -2 & {\bf 1} & -1 & -3 & -3 & -5\\
 \texttt{a} & -8  & -5 & -2 & -1 & {\bf 2} & {\bf 0} & -2 & -2 \\
 \texttt{g} & -10 & -7 & -4 & -3 & 0 & 1 & {\bf 1} & -1 \\
 \texttt{c} & -12 & -7 & -6 & -3 & -2 & 1 & 0 & {\bf 2}  \\\hline
 \end{tabular}
\end{center}

The optimal alignment is then found by {\bf back-tracing} the source of the optimal subscores (i.e. outlined in black), i.e.,

\begin{center}
\begin{tabular}{ccccccc}
T & A & C &  A &  T &  G & C \\
$\mid$ &  & $\mid$ & $\mid$ & & $\mid$ & $\mid$\\
T & T & C & A & $-$ & G & C \\
\end{tabular}
\end{center}


\subsection{Smith-Waterman Algorithm}

So far we have assumed we know which sequences we want to align.
Suppose now that we want to find the optimal alignment between
subsequences of $x$ and $y$. The optimal alignment of subsequences
of two sequences is called the best local alignment.  This algorithm
is nearly the same as Needleman-Wunsch with just a couple
differences.  The first difference is that
\[
F(i,j)=\max\left\{
\begin{array}{l}
0 \\
F(i-1,j-1)+S(x_i,y_i) \\
F(i-1,j)-2 \\
F(i,j-1)-2
\end{array}
\right. .
\]
The other difference is that the trace back step starts at the
highest value of $F$.  From there you trace back in the DP matrix
until the first 0 is found.  This will yield the best local
alignment of $x$ and $y$.


\subsubsection{Smith-Waterman Example}

Using the previous matrix:

\begin{center}
\begin{tabular}{|c|rrrrrrrr|}\hline
              &      &{\tt t}    &{\tt a}      &{\tt c} &{\tt a}     &{\tt t}     &{\tt g} &{\tt c}\\ \hline% ttcac
              & 0   & 0  &0  & 0  & 0  & 0  & 0 & 0\\
 \texttt{t} & 0  & 1 & 0 & 0 & 0 & 1 & 0 & 0\\
 \texttt{t} & 0  & 1 & 0 & 0 & 0 & 1 & 0 & 0\\
 \texttt{c} & 0  & 0 & 0 & {\bf 1} & 0 & 0 & 0 & 1\\
 \texttt{a} & 0  & 0 & 1 & 0 & {\bf 2} & 0 & 0 & 0\\
 \texttt{g} & 0 & 0 & 0 & 0 & 0 & 1 & {\bf 1} & 0 \\
 \texttt{c} & 0 & 0 & 0 & 1 & 0 & 0 & 0 & {\bf 2} \\\hline
 \end{tabular}
\end{center}

Now {\bf back-tracing from the maximum}  (i.e. outlined in black), yields three possible local alignments

\begin{center}
\begin{tabular}{ccccccc}
T & A & C &  A &  T &  G & C \\
 &  & $\mid$ & $\mid$ & & $\mid$ & $\mid$ \\
 &  & $\mid$ & $\mid$ & &  & \\
 &  & & & & $\mid$ & $\mid$  \\
T & T & C & A & $-$ & G & C \\
\end{tabular}
\end{center}

\section{Hidden Markov Models}

A HMM is a statistical model where the system is assumed to be a
Markov process but the state is not observed. However, one does
observe some kind of outcome that comes from being in a state.  We
call this outcome an emission.  We will let $y_1,y_2,\ldots,y_n$ be
a sequence of states which we cannot observe.  We let
$x_1,x_2,\ldots,x_n$ be the sequence of emissions observed
corresponding to $y_1,\ldots,y_n$.  The object is to find out which
state we are in for each emission.

\subsection{Example: Occasionally Dishonest Casino}

In this example, we suppose that a casino usually uses a fair die
but occasionally uses a loaded die.  As is expected, the rolls of a
fair die have the usual probabilities.  The probabilities for the
rolls of the loaded die are $\P (x_i=j\mid y_i=L)=1/10$ for
$j=1,\ldots,5$ and $\P (x_i=6\mid y_i=L)=1/2$.  The emissions for
this example are a sequence of die rolls.  The hidden process is the
die being used. Though we do not observe which die is being used we
assume we know the transition probabilities of the states.  The
transition probabilities are $\P(y_{i+1}=F\mid y_i=F)=.95$,
$\P(y_{i+1}=L\mid y_i=F)=.05$, $\P(y_{i+1}=F\mid y_i=L)=.1$, and
$\P(y_{i+1}=L\mid y_i=L)=.9$. These transition probabilities can be
represented in the Transition matrix
\[
T=\left(
  \begin{array}{cc}
    .95 & .05 \\
    .1 & .9 \\
  \end{array}
\right).
\]
We will assume with probability 1 that we know the casino is using
the fair die before the first roll.

\subsection{Viterbi Algorithm}

The Viterbi algorithm provides one way to solve this problem.  It is
a dynamic programming algorithm that considers all possible routes
to a state but only keeps the most likely one.  We will present the
general algorithm for a finite number of states $s$.  The
assumptions are that we know the transition probabilities of the
state as well as the emission probabilities.  Let $n$ be the number
of emissions.  Let $T_{k,i}=\P(y_{i+1}=i\mid y_{i}=k)$ and
$e_i(x_j)=\P(x_j\mid y_j=i)$.  Let $V$ be a $s\times (n+1)$ matrix
which will be our dynamic programming matrix.  The algorithm is as
follows:
\begin{enumerate}
\item Initialization: Define $V$ as a $s\times (n+1)$ matrix.
\begin{enumerate}
\item $V(1,0)=1$;
\item $V(i,0)=0$ for $i=2,\ldots,s$
\end{enumerate}
\item Recursion: for $j=1,\ldots,n$ and $i=1,\ldots,s$
\begin{enumerate}
\item $V(i,j)=e_i(x_j)\max_k \{V(k,,j-1)T_{k,i}\}$
\item Each time before iterating from $j$ to $j+1$ you must scale the $j$th
column of $V$ so that the column sums to 1.
\item Trace back: record $\argmax_k\{V(k,j-1)T_{k,i}\}$
\end{enumerate}
\item Termination: Start at $\argmax_k\{V(k,n)\}$ and sweep back.
\end{enumerate}

There are a few things to note about this algorithm.  In step 1 we
assumed that we know with probability 1 what the initial state is.
However, this algorithm allows any initial distribution.  In step 3
we find which state we most likely terminate in.  From there we
sweep back by using the information from the trace back step to
decide which state we most likely came from.

\subsection{Forward-Backward Algorithm}

The Viterbi algorithm has advantages and disadvantages.  The
advantage is that it is fast relative to the forward-backward
algorithm.  The disadvantage is that it doesn't yield actual
probabilities but just tells you which state you are most likely in.
Before we derive the forward-backward algorithm we review some
useful concepts from probability theory.

Recall the definition of conditional probability says that for two
events $A$ and $B$
\[
\P(A\cap B)=\P(A\mid B)\P(B).
\]
This can be extended to the intersection of three events $A$, $B$,
and $C$ by
\[
\P(A\cap B\cap C)=\P(A)\P(B\mid A)\P(C\mid B\cap A).
\]
Recall also that for a discrete joint density function $f(x,y)$ the
marginal density function is defined as
\[
f(x)=\sum_y f(x,y)dy.
\]

Now we will derive the forward-backward algorithm.  Let $\bar{x}$ be
a vector containing all the emissions. We wish to find $\P(y_i=k\mid
\bar{x})$.  Using the definition of conditional probability we have
\begin{eqnarray*}
\P(y_i=k\mid \bar{x}) &=& \frac{\P(y_i=k,\bar{x})}{\P(\bar{x})} \\
&=& \frac{\P(y_i=k,x_1,\ldots,x_i)\P(x_{i+1},\ldots,x_n\mid
y_i=k,x_1,\ldots,x_i)}{\P(\bar{x})}
\end{eqnarray*}
Note that $\P(\bar{x})$ is a normalizing constant that you can use
to scale the probabilities so that $\sum_k\P(y_i=k\mid \bar{x})=1$.
We let $f_k(i)=\P(y_i-k,x_1,\ldots,x_i)$ and $b_k(i)=\P(x_{i+1},
\ldots,x_n\mid y_i=k,x_1,\ldots,x_i)$.  We call $f_k(i)$ the forward
equation and $b_k(i)$ the backward equation.  This gives us
\[
\P(y_i=k\mid \bar{x})=\frac{f_k(i)b_k(i)}{\P(\bar{x})}.
\]

\subsubsection{Forward Algorithm}

We will now derive an expression for $f_k(i)$.  Using the definition
of a marginal density function and conditional probability for
multiple events we have that
\begin{eqnarray*}
f_k(i) &=& \P(y_i=k,x_1,\ldots,x_i) \\
&=& \sum_{j=1}^s \P(y_i=k,x_1,\ldots,x_i,y_{i-1}=j) \\
&=& \sum_{j=1}^s \P(x_i\mid
y_i=k,x_1,\ldots,x_{i-1},y_{i-1}=j)\P(y_i=k,x_1,\ldots,x_{i-1},y_{i-1}=j)
\end{eqnarray*}

Since we are assuming the underlying process $y_1,\ldots,y_n$ is a
markov process we can say that
\[
\P(x_i\mid y_i=k,x_1,\ldots,x_{i-1},y_{i-1}=j)=\P(x_i\mid y_i=k).
\]
This gives us
\begin{eqnarray*}
f_k(i) &=& \sum_{j=1}^s \P(x_i\mid
y_i=k)\P(y_i=k,x_1,\ldots,x_{i-1},y_{i-1}=j) \\
&=& \sum_{j=1}^s e_k(x_i)\P(y_i=k\mid
x_1,\ldots,x_{i-1},y_{i-1}=j)\P(x_1,\ldots,x_{i-1},y_{i-1}=j) \\
&=& \sum_{j=1}^s e_k(x_i)\P(y_i=k\mid
,y_{i-1}=j)\P(x_1,\ldots,x_{i-1},y_{i-1}=j) \\
&=& e_k(x_i)\sum_{j=1}^s T_{j,k} f_j(i-1)
\end{eqnarray*}

Now that we have a recursive equation for $f_k(i)$ we have an
algorithm for the forward equation.
\begin{enumerate}
\item Initialization: Define a $s\times(n+1)$ matrix $F$ where
$F_{i,j}=f_i(j)$.
    \begin{enumerate}
    \item $f_1(1)=1$
    \item $f_k(1)=0$ for $k=2,\ldots,s$
    \end{enumerate}
\item Recursion: $\hat{f}_k(j)=e_k(x_i)\sum_{i=1}^s T_{i,k}f_i(j-1)$
for $k=1,\ldots,s$ and $j=1,\ldots,n$
    \begin{enumerate}
    \item Let $s(j)=\sum_{i=1}^s \hat{f}_i(j)$, which is our scaling
    factor.  Before moving to the next column of the matrix by
    iterating $j$ let $f_k(j)=\hat{f}_k(j)/s(j)$ for $k=1,\ldots,s$.
    \end{enumerate}
\end{enumerate}

\subsubsection{Backward Algorithm}

Now we derive a recursive equation for the backward equation
$b_k(i)$.  The derivation is very similar to the forward equation.

\begin{eqnarray*}
b_k(i) &=& \P(x_{i+1},\ldots,x_n\mid y_i=k) \\
&=& \sum_{j=1}^s \P(x_{i+1},\ldots,x_n,y_{i+1}=j \mid y_i=k) \\
&=& \sum_{j=1}^s \P(x_{i+1}\mid
x_{i+2},\ldots,x_n,y_{i+1}=j,y_i=k)\P(x_{i+2},\ldots,x_n,y_{i+1}=j\mid
y_i=k) \\
&=& \sum_{j=1}^s \P(x_{i+1}\mid
y_{i+1}=j)\P(x_{i+2},\ldots,x_n,y_{i+1}=j\mid y_i=k) \\
&=& \sum_{j=1}^s e_j(x_{i+1})\P(x_{i+2},\ldots,x_n\mid y_{i+1}=j,y_i=k)\P(y_{i+1}=j\mid y_i=k) \\
&=& \sum_{j=1}^s e_j(x_{i+1})b_j(i+1)T_{k,j}
\end{eqnarray*}

This yields the following algorithm for the backward equation.

\begin{enumerate}
\item Initialization:  Define $s\times n$ matrix $B$ with elements
$B_{i,j}=b_i(j)$
    \begin{enumerate}
    \item $b_k(n)=T_{k,1}$ for $k=1,\ldots,s$
    \end{enumerate}
\item Recursion: Let $\hat{b}_k(j)=\sum_{i=1}^s e_i(x_{j+1})b_i(j+1)T_{k,j}$ for $j=n-1,\ldots,1$ and $k=1,\ldots,s$
    \begin{enumerate}
    \item Let $s(j)=\sum_{k=1}^s \hat{b}_k(j)$ and before each time you
    iterate $i$ make sure to scale it by letting
    $b_k(j)=\hat{b}_k(j)/s(j)$.
    \end{enumerate}
\end{enumerate}

\subsection{Forward-Backward Algorithm for Transitions}

Suppose that we are trying to estimate the states of a HMM but we
don't know the matrix $T$ or the emission probabilities $e_k(x_j)$
or both. This alternative forward-backward algorithm along with the
Baum-Welch algorithm, which is presented later, is useful for
estimating these probabilities.  We start with an initial guess for
your transition and emission probabilities.  After several
iterations of the forward-backward algorithm the Baum-Welch
algorithm will help us converge to the true probabilities. We let
\[
a_{m,k}(j)=\P(y_j=k,y_{j-1}=m\mid
\bar{x})=\frac{f_{m,k}(j)b_{m,k}(j)}{\P(\bar{x})}
\]
for $j=2,\ldots,n$; $m=1,\ldots,s$ and $k=1,\ldots,s$.

\begin{enumerate}
\item Initialization: Let $F$ be an $s^2\times n$ matrix where
$F_{i,j}=f_{m,k}(j)$.  The reason for having $s^2$ rows is that at
each time $j$ we must consider all possible state transitions.  For
example, a HMM with 2 states we may wish to let
$F_{1,j}=f_{1,1}(j)$, $F_{2,j}=f_{1,2}(j)$, $F_{3,j}=f_{2,1}(j)$,
and $F_{4,j}=f_{2,2}(j)$.
\begin{enumerate}
\item $f_{1,1}(1)=1$ and $f_{m,k}(1)=0$ for $m,k>1$.
\end{enumerate}
\item Recursion: for $i=2,\ldots,n$; $m=1,\ldots,s$ and
$k=1,\ldots,s$ set
\[
f_{m,k}(j)=e_k(x_j)T_{m,k}\sum_{i=1}^s f_{i,m}(j-1)
\]
\begin{enumerate}
\item Remember to scale.
\end{enumerate}
\end{enumerate}

It is left to the reader to derive this algorithm as well as the
backward algorithm in the exercises.  You will need to show that
$b_{m,k}(j)=b_k(j)$ and that the backward equation is in fact the
same as the backward equation for $\P(y_i=k \mid \bar{x})$.

\subsubsection{Baum-Welch (Expectation-Maximation) Algorithm}
This algorithm uses the forward-backward algorithm for transitions
to estimate $T$ and $e_k(x_j)$.

\begin{enumerate}
\item Initialization: Choose or guess initial values for your
transition probabilities.
\item E-step: Apply a forward-backward algorithm for transitions to
estimate $a_{m,k}(j)$ for $j=2,\ldots,n$; $m=1,\ldots,s$ and
$k=1,\ldots,s$.  Note that the sequence state probabilities, denoted
$p_k(j)$, can be obtained from
\[
\hat{p}_k(j)=\P(y_i=k\mid \bar{x})=\sum_{m=1}^s
\P(y_i=k,y_{i-1}=m\mid \bar{x})=\sum_{m=1}^s \hat{a}_{m,k}(j)
\]
where $\hat{a}_{m,k}(j)$ and $\hat{p}_k(j)$ represent estimates of
$a_{m,k}(j)$ and $p_k(j)$ respectively.  These estimates are based
on your guesses for $T$ and $e_k(x_j)$.
\item M-step: Estimate the emission and transition probabilities.
    \begin{enumerate}
    \item Loaded die emissions: Letting $I(x_i=t)=1$ if emission
    $x_i=t$ and 0 otherwise, then
    \[
    \hat{e}(t)=\frac{\sum_{i=1}^n\hat{p}_k(i)I(x_i=t)}{\sum_{i=1}^n\hat{p}_k(i)}
    \]
    for $t=1,\ldots,6$.
    \item Transitions:
    \[
    \hat{T}_{m,k}=\frac{\sum_{i=2}^n\hat{a}_{m,k}(i)}{n-1}
    \]
    \end{enumerate}
\item Repeat steps 2-3 until convergence.
\end{enumerate}

\section{Exercises}

\begin{problem}
Download http://statistics.byu.edu/johnson/DNAsequences.fa.  This
file contiains two long DNA sequences, one from the \textit{C.
elegans pha4} gene and the other is from the \textit{C. Briggsae}
version of the gene (both species are nematode worms).  Note that
both sequences are very long (~5000 bases).  Use Needleman-Wunsch
algorithm to align sequences int he file.  Print the DP matrix to a
file, and display your aligned result in a text file containing
three lines in the following format:

\begin{tabular}{ccccccccccc}
- & C & A & G & A & T & G & A & G & C & A \\
  & $\mid$ & $\mid$ & & $\mid$ & $\mid$ & $\mid$ & $\mid$ & & $\mid$ & $\mid$ \\
T & C & A & - & A & T & G & A & A & C & A
\end{tabular}

where `-' indicates a gap, `$\mid$' indicates that the sequences are
the same for the base and leave a blank space on the second line for
the bases that are aligned but are not the same.  Turn in the
dynamic programming matrix file and the sequence alignment file for
this problem
\end{problem}

\begin{problem}
Using the same DNA sequences as in the previous problem, align the
sequences using the Smith-Waterman local alignment algorithm.  Turn
in the DP matrix and the sequence alignment as in the previous
problem.  What differences do you observe between the two
alignments?
\end{problem}

\begin{problem}
Download http://statistics.byu.edu/johnson/dierolls.txt.  This file
contains two lines of data from the `Occasionally Dishonest Casino'
example from class.  The first line contains the die rolls.  The
second line is the (hidden) Markov state process (which die was used
Fair/Loaded).  Using only die rolls and the transition and emission
probabilities form class, apply the Viterbi algorithm to estimate
the most likely sequence of the hidden Markov Chain.  Print the DP
matrix to a file.  Also print both your estimated Markov state
sequence as well as the actual sequence to a file.  Compare and
comment on how the algorithm did in estimating the Markov Chain.
Turn in you DP matrix file, your state sequence file, and your brief
comments.
\end{problem}

\begin{problem}
Using the dieroll data at
http://statistics.byu.edu/johnson/dierolls.txt, apply a
forward-backward algorithm to estimate the die state probabilities.
Plot this result as well as your Viterbi result and compare the
results.  Comment on what you observe.
\end{problem}

\begin{problem}
Download the die roll data set from
http://statistics.byu.edu/johnson/dierolls2.txt.  The data set comes
from an `Occasionally Dishonest Casino' example where the emission
probabilities for the loaded die are not known.  Assume you know the
fair die probabilities and the transition probabilities are the same
as given in the example above.  Use an iterative Baum-Welch (EM)
algorithm to estimate both the loaded die emission probabilities as
well as the state probabilities (Hint: Start with a guess for the
loaded die emissions, apply a forward-backward, estimate emissions,
apply forward-backward, etc., until convergence).  Write your
forward, backward, probability and emission results to a file and
turn this in.
\end{problem}

\begin{problem}
Derive the recursion for the forward-backward algorithm for
estimating
\[
\P(y_i=m,y_{i-1}=k\mid \bar{x}).
\]
You will be given the complete forward-backward algorithm in lab, so
deriving the recursion is sufficient for this problem.
\end{problem}

\begin{problem}
Download the die roll data set from
http://statistics.byu.edu/johnson/dierolls3.txt.  The data set comes
from an `Occasionally Dishonest Casino' example where the emission
probabilities for the loaded die and the transitions are not known.
Assume you know the fair die probabilities.  Use an iterative
Baum-Welch (EM) algorithm to estimate the loaded die emission
probabilities, the transition probabilities and the state
probabilities.  Write your forward, backward, probability,
transition, and emission results to a file and turn this in.
\end{problem}


\end{document}
